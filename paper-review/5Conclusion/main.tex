\section{Conclusion}
We saw how the sequential NFA execution bottleneck on the Micron Automata Procesor (AP). We identify two main
challenges to applying enumerative NFA parallelization techniques
on the AP: (1) high state-tracking overhead for input composition (2)
huge computational complexity for enumerating parallel paths on
large NFAs. Using the AP flow abstraction and properties of FSMs
like small input symbol transition range, connected components and
common parents we amortize the overhead of state-tracking and
realize a time-mutliplexed execution of enumerated paths. To tackle
the computational complexity, we leverage properties of the FSMs
like path convergence to dynamically reduce the number of executed
enumerated flows.\\
We can make a conclusion that most of the times , special-oriented hardware outperforms general-oriented if it is well leveraged. Also, we saw that special-oriented hardware with a special management outperforms the same common special-oriented hardware.
This conclusion may look trivial , but it's just a reminder of how \textit{Subranamiyan} \cite{} implemented the FSM parallelization in the AP with outstanding results.