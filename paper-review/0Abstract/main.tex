\begin{abstract}
    
    A Finite State Machine (FMS) is an important abstraction for solving several problems.
    This computational model  is widely used  for many application domains, such as
    matching regular-expression, tokenizing and Huffman decoding.
    These  embarrassingly sequential applications with irregular memory access patterns perform
    perform poorly on conventional von-Neumann architectures. The Micron Automata
    Processor (AP) is an in-situ memory-based computational architecture that accelerates non-deterministic finite automata (NFA) 
    processing in hardware. However, each FSM on the AP is processed
    sequentially, limiting potential speedups.
    We explore the FMS parallelization problem in context of the AP. The classical parallelization
    techniques to NFAs executing on AP is non-trivial because of high state-transition
    tracking overheads and exponential computation complexity.
    We take a special look to a work that aims to improve performance by custom parallelization
    of FSM processing on AP.
    
    
\end{abstract}